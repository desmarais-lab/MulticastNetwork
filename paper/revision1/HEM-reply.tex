\documentclass[12pt]{article}

\topmargin 0pt \advance \topmargin by -\headheight \advance
\topmargin by -\headsep \textheight 8.9in \oddsidemargin 0pt
\evensidemargin \oddsidemargin \marginparwidth 0.5in \textwidth
6.5in

\renewcommand{\baselinestretch}{1.33}

\usepackage{amsmath}
\usepackage{amssymb}
\usepackage{amsthm}
\usepackage{exscale}
\usepackage{multirow}
\usepackage[mathscr]{eucal}
\usepackage{bm}
\usepackage{eqlist} % Makes for a nice list of symbols.
\usepackage[final]{graphicx}
\usepackage[dvipsnames]{color}
\usepackage{verbatim}
\usepackage{natbib}
\usepackage{caption}
\usepackage{enumerate}
\usepackage{subcaption}
\usepackage{algorithm}
\usepackage{algpseudocode}
\usepackage{booktabs}

\usepackage{etoolbox}

\newtheorem{thm}{Theorem}

\newcommand{\aset}{{\mathcal A}}
\newcommand{\boldbeta}{{\boldsymbol{\beta}}}

\def\red{\color{red}}

\theoremstyle{definition}
\newtheorem{definition}{Definition}

\def\sgn{\mathrm{sign}}
\def\E{\mathrm{E}}
\def\Var{\mathrm{Var}}
\def\pr{\mathrm{Pr}}
\def\tr{\mathrm{tr}}
\def\ve{\mathrm{vec}}
\def\diag{\mathrm{diag}}

\newcommand{\bff}{\mathrm{\bf f}}
\def\aset{\mathcal{A}}
\def\bset{\mathcal{B}}
\def\sset{\mathcal{S}}
\def\ba{\boldsymbol{a}}
\def\hba{\boldsymbol{\hat a}}
\newcommand{\hf}{\widehat \bff}
\newcommand{\bphi}{\mbox{\boldmath $\phi$}}
\newcommand{\hphi}{\widehat\bphi}
\newcommand{\Sig}{\mathbf{\Sigma}}
\newcommand{\hSig}{\widehat\Sig}
\newcommand{\hbSig}{\widehat\bSigma}
\newcommand{\hlam}{\widehat\lambda}
\newcommand{\hLam}{\widehat \bLam}
\newcommand{\bLam}{\mbox{\boldmath $\Lambda$}}
\def\br{\boldsymbol{r}}
\def\hbr{\boldsymbol{\hat r}}
\def\bu{\boldsymbol{u}}
\def\bv{\boldsymbol{v}}
\def\bx{\boldsymbol{x}}
\def\bxi{\boldsymbol{\xi}}
\def\by{\boldsymbol{y}}
\def\bZ{\boldsymbol{Z}}
\def\bz{\boldsymbol{z}}
\def\bp{\boldsymbol{p}}
\def\bX{\boldsymbol{X}}
\def\bW{\boldsymbol{W}}
\def\bY{\boldsymbol{Y}}
\def\bQ{\boldsymbol{Q}}
\def\hbQ{\boldsymbol{\hat Q}}
\def\bA{\boldsymbol{A}}
\def\hbA{\boldsymbol{\hat A}}
\def\bD{\boldsymbol{D}}
\def\hbD{\boldsymbol{\hat D}}
\def\bI{\boldsymbol{I}}
\def\bS{\boldsymbol{S}}
\def\bV{\boldsymbol{V}}
\def\bR{\boldsymbol{R}}
\def\hbR{\boldsymbol{\hat R}}
\def\tbR{\boldsymbol{\tilde R}}
\def\bH{\boldsymbol{H}}
\def\bL{\boldsymbol{L}}
\def\bM{\boldsymbol{M}}
\def\bN{\boldsymbol{N}}
\def\boldf{\boldsymbol{f}}

\def\bpi{\boldsymbol{\pi}}
\def\bmu{\boldsymbol{\mu}}
\def\hbmu{\boldsymbol{\hat\mu}}
\def\bzero{\boldsymbol{0}}
\def\bSigma{\boldsymbol{\Sigma}}
\def\hbSigma{\boldsymbol{\hat\Sigma}}
\def\bOmega{\boldsymbol{\Omega}}
\def\bPsi{\boldsymbol{\Psi}}
\def\bDelta{\boldsymbol{\Delta}}

\def\bG{\boldsymbol{\Gamma}}
%\def\hbGamma{\boldsymbol{\hat\Gamma}}
%\def\bgamma{\boldsymbol{\gamma}}
%\def\hbgamma{\boldsymbol{\hat\gamma}}

\def\bGamma{\boldsymbol{\Sigma}}
\def\hbGamma{\boldsymbol{\hat\Sigma}}
\def\bgamma{\boldsymbol{\sigma}}
\def\hbgamma{\boldsymbol{\hat\sigma}}

\def\bbeta{\boldsymbol{\beta}}
\def\bpsi{\boldsymbol{\psi}}
\def\btheta{\boldsymbol{\theta}}
\def\mvnorm{N_k(\bzero,\mathbf{\Sigma})}

\def\F{\mathcal{F}}
\def\G{\mathcal{G}}
\def\H{\mathcal{H}}
\def\P{\mathcal{P}}
\def\U{\mathcal{U}}

\def\be{\begin{equation}}
\def\ee{\end{equation}}
\def\ben{\begin{equation*}}
\def\een{\end{equation*}}
\def\bea{\begin{eqnarray}}
\def\eea{\end{eqnarray}}
\def\bean{\begin{eqnarray*}}
\def\eean{\end{eqnarray*}}

\date{\today}
\begin{document}
%\begin{flushright}
%March 31, 2012
%\end{flushright}


\begin{center}
\textbf{Authors' Responses to Associate Editor}
\end{center}
\begin{center}
\textsl{The Hyperedge Event Model}
\end{center}

Thank you for acknowledging our efforts and contributions, and also for your constructive suggestions, which are very helpful to improve the quality of our paper. 

\begin{itemize}
	\item Presentation

	\textbf{  Response:} 
	\begin{itemize}
		\item Unusual notations such as actor $A$ and covariates $y$. Also $u_{ie}$ is the $i$th line of matrix $u_e$, while $\tau_e = min_i(\tau_{ie})$.  \textcolor{blue}{(done)}
		\item Change the order of Equation (2.2) and (2.1) and not use `intensity'. \textcolor{blue}{(done)}
		\item Hard to understand 2.2 without 2.3. More explanation on $\tau_{ie}$ above Equation (2.5) and what $\mu$ represents in $V(\mu)$. \textcolor{blue}{(done)}
	\end{itemize}

	\item Discussion of relevant literature

	\textbf{  Response:} 
	\begin{itemize}
		\item Perry and Wolfe (2013) arxiv version has a model for multicast. Any differences/advantages? \textcolor{blue}{(done)}
		\item Why cite Snijders (1996) in 2.3? Be specific. \textcolor{blue}{(done)}
	\end{itemize}
	
	\item Model, covariates and missing data
	
		\textbf{  Response:} 
		\begin{itemize}
			\item Observations $(s_e, r_e, t_e)_{e=1,\ldots,E}$ are not conditionaly independent since covariates depend on last 7 days. State this in Section 2 and modify out-of-sample algorithm using $(s_{e'}, r_{e'}, t_{e'})_{e':t_e<t_e'<t_e+l_e}$.  \textcolor{red}{(discuss)} 
		\end{itemize}
	\item MCMC sampler
	
		\textbf{  Response:} 
		\begin{itemize}
			\item Details on M-H proposals for $\boldsymbol{b}$ and $\boldsymbol{\eta}$ in Section 3.2. \textcolor{blue}{(done)}
			\item Inefficient sampler for $u_{iej}$ especially when most are one-to-one. Comment on this and the mixing of MCMC samplers.  \textcolor{red}{(discuss)} 
			\item Move Geweke to appendix \textcolor{blue}{(done)} and use larger number of nodes and events.
			\item Computational complexity per iterations of the samplers.
		\end{itemize}
		
	\item Typos \textcolor{blue}{(done)}
	
		\textbf{  Response:} We fixed all the typos identified by the reviewer as well as other writing issues, and we highly appreciate your considerable comments on these which were extremely helpful.
\end{itemize}



\begin{center}

\textbf{Authors' Responses to Reviewer 1}
\end{center}
\begin{center}
\textsl{The Hyperedge Event Model}
\end{center}

Thank you for acknowledging our efforts and contributions, and also for your constructive suggestions, which are very helpful to improve the quality of our paper.
\begin{itemize}
	\item Presentation and writing
	
	\textbf{  Response:} We fixed all the typos, unclear parts, and issues in the bibliography \textcolor{blue}{(bib not working)} identified by the reviewer. We highly appreciate your considerable comments on these which were extremely helpful. \textcolor{blue}{(done)}

  \item Literature review

\textbf{  Response:}  We added a subsection following the description of the HEM in which we ground it in the structure of existing models for networks. 
\begin{itemize}
	\item More comprehensive review including temporal ERGMS and dynamic latent variable models, and discuss contributions and novelties in the light of alternatives
\end{itemize}

\item Section 2

\textbf{  Response:} Rewrite Section 2 to provide a much clear picture of the model.


\item Prior specification

\textbf{  Response:} 
\begin{itemize}
	\item Use weakly informative priors as generic priors instead of assuming $N(0, \infty)$  \textcolor{blue}{(done)}
	\item Sensitivity analyses to check how much posterior inference is affected by the hyperparameters' settings, and, possibly, suggest some default values.  \textcolor{red}{(discuss)} 
\end{itemize}

\item Posterior computation  \textcolor{red}{(discuss)} 

\textbf{  Response:} 
\begin{itemize}
	\item Type of MH, proposal distribution, acceptance rate, smart proposal
	\item Comment on poor mixing on data augmentation
	\item Extent of scaling, bigger dataset, information on computational time
\end{itemize}
\item Application  \textcolor{red}{(discuss)} 

\textbf{  Response:} 
\begin{itemize}
	\item Better baseline than random guess $1/18$ 
	\item Compare with SAOMs and extensions in PPE and PPC 
	\item Bad results in predicting timestamps (MdAPE)  \textcolor{blue}{(done)}
	\item More conservative interpretations \textcolor{blue}{(done)}
\end{itemize}
\end{itemize}

\newpage
\begin{center}
	
	\textbf{Revision Plans}
\end{center}
\begin{enumerate}
	\item Section 1: Literature review
	\begin{itemize}
		\item Bomin: add literature review on Perry and Wolfe (2013) arxiv version and comment how our model differs from point process based models (AE1 bullet 2)
		\item Bruce: add literature review on the general class of dynamic network inference including temporal ERGMS and dynamic latent variable models (R1 comment 2)
	\end{itemize}
	\item Section 2: Generative process
		\begin{itemize}
			\item Bomin: change few notations and add notation table in Appendix (AE1 bullet 1)
			\item Bruce: overall rewriting such as rephrasing or clarification (R1 comment 3 \& AE1 bullet 1, minor ones already resolved)
		\end{itemize}
			\item Section 3: Inference
			\begin{itemize}
		\item Bomin: add more MH details and add a subsection 3.2 for computational issues---e.g., complexity and limitations (R1 comment 4, 5 \& AE1 bullet 4)
				\item Bruce: check the added subsection 3.2 and revise
			\end{itemize}
			\item Section 4: Application
			\begin{itemize}
				\item Bomin: re-run PPE considering conditional dependence (section 4.2) and update results \& interpretations in 4.2 \& 4.4 (R1 comment 6 \& AE1 bullet 2)
				\item Bruce: add why direct comparison with SAOMs not possible and come up with better idea than random guess...? (R1 comment 6)
			\end{itemize}			
	
				\item Section 5: Conclusion
				\begin{itemize}
					\item Bruce: possibly further discussing our contribution in the light of alternatives added in the literature review (R1 comment 2)
				\end{itemize}			
					
					\item Bibliography
					\begin{itemize}
						\item Bomin: tons of issues but somehow changes are not reflected...? Double check! (R1 comment 1)
					\end{itemize}	
					
					\item Discuss: sensitivity analysis?					
\end{enumerate}

\end{document}

