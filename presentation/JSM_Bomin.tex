\documentclass{beamer}
\usepackage{graphicx}
\usepackage{tikz}
\usepackage{mathtools}
\usepackage{ragged2e}
\usepackage{adjustbox}
\title{The Hyperedge Event Model}
\author{Bomin Kim\\ \vspace{0.25cm} Department of Statistics\\Pennsylvania State University}
\date{July 29, 2018\\\vspace{0.5cm} Joint Statistical Meetings 2018}
\begin{document}
\maketitle
\begin{frame}
		\frametitle{Collaborators}
		\hspace{0.5cm}
		\begin{figure}
								{\includegraphics[width=.3\textwidth]{aaron.jpg}}
			{\includegraphics[width=.2245\textwidth]{Bruce.jpeg}}
				{\includegraphics[width=.3\textwidth]{Hanna.jpeg}}
		\end{figure}
\begin{itemize}
	\item \textcolor{blue}{Aaron Schein}\\\footnotesize{College of Information and Computer Sciences, UMass Amherst}
	\normalsize
	\item \textcolor{blue}{Bruce Desmarais}\\\footnotesize{Department of Political Science, Pennylvania State University}
\item		\normalsize\textcolor{blue}{Hanna Wallach}\\ \footnotesize{Microsoft Research NYC}	\normalsize
		\end{itemize}
\end{frame}
\begin{frame}
		\frametitle{Motivations: \textcolor{red}{H}yperedge \textcolor{red}{E}vent \textcolor{red}{M}odel}
		\begin{itemize}
			\item \textcolor{red}{H}\textcolor{blue}{yperedge}\\
			edges including one sender and multiple receivers or one receiver and multiple
senders.
			\item \textcolor{red}{E}\textcolor{blue}{vent}\\
			timestamped events
			\item  \textcolor{red}{M}\textcolor{blue}{odel}\\
			statistical framework to jointly \\
		\end{itemize}
		\centering\Large
		``who interacts with whom, and when?"
\end{frame}

\begin{frame}
	\frametitle{Generative Process: ``\textcolor{red}{Who} Interacts with \textcolor{red}{Whom}"}
For each edge or event $e = 1,\ldots, E,$ 	\normalsize\vspace{0.15cm}
	\begin{itemize}
		\item \textcolor{blue}{Receiver intensity} for every sender-receiver pair $(i, j)_{i\neq j}$
		\begin{equation*}
		\lambda_{iej} = \boldsymbol{b}^T \boldsymbol{x}_{iej},
		\end{equation*}
		where $\boldsymbol{x}_{iej}$ is a set of receiver selection features or covariates. \vspace{0.15cm}
		\item Every sender selects \textcolor{red}{candidate receivers} from non-empty \textcolor{blue}{multivariate Bernoulli distribution} $\boldsymbol{u}_{ie} \sim \mbox{MB}_G ({\lambda}_{ie1},\ldots, {\lambda}_{ieA})$
				\begin{equation*}
			P(\boldsymbol{u}_{ie}|\boldsymbol{b}, \boldsymbol{x}_{iej}) \propto \exp\Big(\log(I(||\boldsymbol{u}_{ie}||_1 > 0))+\sum_{j\neq i} \lambda_{iej}u_{iej}\Big)
				\end{equation*}
	\end{itemize}
	
\end{frame}
\begin{frame}
	\frametitle{Generative Process: ``and \textcolor{red}{When}"}
		\begin{itemize}
			\item \textcolor{blue}{Timing rate} for each sender
			\begin{equation*}
			\mu_{ie} = g^{-1}(\boldsymbol{\eta}^T \boldsymbol{y}_{ie})
			\end{equation*}
			\item \textcolor{blue}{Generalized linear model} (GLM) such that time increment $\tau_{ie}$ satisfy
			\begin{equation*}
			E(\tau_{ie}) = 	\mu_{ie} \mbox{ and } V(\tau_{ie}) = V(	\mu_{ie})
			\end{equation*}
			\item Select the sender-receiver-set with \textcolor{blue}{the smallest time increment}
						\begin{equation*}
						\begin{aligned}
						s_e &= \mbox{argmin}_{i}(\tau_{ie}),\\
						\boldsymbol{r}_e &= \boldsymbol{u}_{s_e e},\\
						t_e &=t_{e-1} + \tau_{s_e e}.
						\end{aligned}
						\end{equation*}
		\end{itemize}
	\end{frame}
	\begin{frame}
		\frametitle{Generative Process: \textcolor{red}{Sender}, \textcolor{red}{Receivers}, and \textcolor{red}{Timestamps}}
					\begin{figure}[H]
						\centering
						\includegraphics[width=1\textwidth]{diagramnew.png}	
						\label{figure:diagram}
					\end{figure}						
			\end{frame}
			
				\begin{frame}
			\frametitle{Application: Montgomery County Government
\textcolor{red}{Email} Data}
\begin{itemize}
				\item Coefficients for receiver selection features
			\item Coefficients for event timing features
			\end{itemize}
		\end{frame}				
		\begin{frame}
			\frametitle{Results: Exploratory Analysis}
			\begin{itemize}
				\item Coefficients for receiver selection features
				\item Coefficients for event timing features
			\end{itemize}
			
		\end{frame}				
				\begin{frame}
					\frametitle{Comparison: Lognormal vs. Exponential}
					\begin{itemize}
						\item Covariates
					\end{itemize}
				\end{frame}	
							\begin{frame}
								\frametitle{Conclusions}
								\begin{itemize}
									\item Covariates
								\end{itemize}
							\end{frame}	
							
\end{document}